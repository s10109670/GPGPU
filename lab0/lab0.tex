\documentclass[12pt,a4paper]{article}

% \newcommand*{\TypeChinese}{} % Chinese support
\newcommand*{\AdvancedDocument}{} % include code and math
\newcommand*{\WithHeader}{}

% basic packages
\usepackage[margin=2cm]{geometry}
\usepackage{graphicx,subfigure,indentfirst,hyperref,colortbl,caption,cite,color,xcolor}
\hypersetup{colorlinks=true,urlcolor=blue,linkcolor=blue}

\ifdefined\AdvancedDocument
	% minted is better than listing.
	\usepackage{minted}
	% it requires minted of a newer version.
	\setminted{linenos=true, frame=lines, framesep=2mm}
	\usepackage{amsmath,amssymb,bm}
\fi

\ifdefined\TypeChinese
	\usepackage{xeCJK,fontspec}
	\XeTeXlinebreaklocale "zh"
	\XeTeXlinebreakskip = 0pt plus 1pt
	\setmainfont{KaiGen Gothic TW}
	\setCJKmainfont{KaiGen Gothic TW}
	\setmonofont{Droid Sans Mono}
	\renewcommand{\baselinestretch}{1.3}
\fi

\ifdefined\WithHeader
	\usepackage{fancyhdr}
	\fancypagestyle{plain}{
		\fancyhf{}
		\chead{GPU Programming 2017 Spring \textbar ~CSIE Department, National Taiwan University}
		\cfoot{\thepage}
		\rfoot{GPGPU Assignment \#0}
	}
	\pagestyle{plain}
	\renewcommand{\headrulewidth}{1pt}
	\renewcommand{\footrulewidth}{2pt}
\fi

\newcommand{\figref}[1]{Figure \ref{Fig:#1}.}
\newcommand{\tabref}[1]{Table \ref{Tab:#1}.}
% \graphicspath{{fig/}}

\begin{document}
\title{GPGPU Assignment \#0}
\author{TA: Yu Sheng Lin \and Instructor: Wei Chao Chen}
\maketitle

\section{Goals}

You have to

\begin{enumerate}
\item Get your OS/IDE/editor configured and be ready to write CUDA code (How? Google is your friend).
\item Become familiar with CUDA syntax (You should have learnt some of it during the first lecture).
\end{enumerate}

\section{Requirements}

In this assignment you have to draw something in text by CUDA.
We have provided a skeleton and some utilization functions (\verb+SyncedMem<T>+ and \verb+MemoryBuffer<T>+).

We allocate a buffer of size $40\times 12$ but one linebreak is required for each line
, so the actual drawing area is $39\times 12$ including the boundary, which is consists of colons.

Here we show a possible output.
\begin{listing}
\begin{minted}{text}
:::::::::::::::::::::::::::::::::::::::
:                                     :
:                                     :
:                                     :
:                                     :
:                 ####          <|    :
:               ######           |    :
:             ########           |    :
:           ##########           |    :
:         ############           |    :
:       ##############           #    :
:::::::::::::::::::::::::::::::::::::::
\end{minted}
\caption{The famous scene in Nintendo Super Mario.}
\end{listing}

\section{Submission}
\begin{itemize}
\item The submission deadline is the midnight on 3/1 Wed. (namely before 3/2).
\item You will be officially registered to this course only if you complete and submit a working solution in time.
\item We will clone your code through Git using script; you may continue to revise your code before the deadline, but we will use the last revision before the deadline.
\item Use a non-public Git repository such as Bitbucket, and make sure that your code can be cloned by these accounts: \url{https://bitbucket.org/johnjohnlys/} or \url{https://github.com/johnjohnlin}.
\item Please fill your information and git repo in \href{https://goo.gl/forms/1R7p6QRMlrnKuImu1}{this form}. TA will test your Git URL one day before the deadline, so you can modify your URL in time if it doesn't work.
\item You should complete this homework by yourself. Do not plagiarize, and do not facilitate plagiarism.  Make sure your Git repository is not accessible by your classmates.
\end{itemize}

Please keep the directory structure of the repo we have provided. For this assignment, we will only judge \verb+lab0/main.cu+.

\section{Hints}

\begin{itemize}
\item \verb+"SyncedMemory.h"+ is under the directory \verb+utils/+.
\item C++11 is required throughout all of the assignment. AFAIK, Visual Studio 2013 and g++ 4.8 or later is recommended.
\item Do not spend much time on optimization or fancy functionalities. This assignment is just for qualification and will not be used for grading.
\end{itemize}

\end{document}
